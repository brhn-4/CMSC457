%You can leave alone everything before Line 84.
\documentclass{article}
\usepackage{url,amsfonts, amsmath, amssymb, amsthm}
\usepackage{pict2e}
\usepackage[percent]{overpic}
% Page layout
\setlength{\textheight}{8.75in}
\setlength{\columnsep}{2.0pc}
\setlength{\textwidth}{6.5in}
\setlength{\topmargin}{0in}
\setlength{\headheight}{0.0in}
\setlength{\headsep}{0.0in}
\setlength{\oddsidemargin}{0in}
\setlength{\evensidemargin}{0in}
\setlength{\parindent}{1pc}
\newcommand{\shortbar}{\begin{center}\rule{5ex}{0.1pt}\end{center}}
%\renewcommand{\baselinestretch}{1.1}
% Macros for course info
\newcommand{\courseNumber}{CMSC/PHYS 457}
\newcommand{\courseTitle}{Introduction to Quantum Computing}
\newcommand{\semester}{Spring 2023}
% Theorem-like structures are numbered within SECTION units
\theoremstyle{plain}
\newtheorem{theorem}{Theorem}[section]
\newtheorem{lemma}[theorem]{Lemma}
\newtheorem{corollary}[theorem]{Corollary}
\newtheorem{proposition}[theorem]{Proposition}
\newtheorem{statement}[theorem]{Statement}
\newtheorem{conjecture}[theorem]{Conjecture}
\newtheorem{fact}{Fact}
%definition style
\theoremstyle{definition}
\newtheorem{definition}[theorem]{Definition}
\newtheorem{example}{Example}
%\newtheorem{problem}{Problem}
\newtheorem{exercise}{Exercise}
\newtheorem{algorithm}{Algorithm}
%remark style
\theoremstyle{remark}
\newtheorem{remark}[theorem]{Remark}
\newtheorem{reduction}[theorem]{Reduction}
%\newtheorem{question}[theorem]{Question}
\newtheorem{question}{Question}
%\newtheorem{claim}[theorem]{Claim}
%
% Proof-making commands and environments
\newcommand{\beginproof}{\medskip\noindent{\bf Proof.~}}
\newcommand{\beginproofof}[1]{\medskip\noindent{\bf Proof of #1.~}}
\newcommand{\finishproof}{\hspace{0.2ex}\rule{1ex}{1ex}}
\newenvironment{problem}[1]{\medskip\noindent{\bf Problem #1.~}}{\shortbar}
\newenvironment{solution}[1]{\medskip\noindent{\bf Solution #1.~}}{\shortbar}
\newcommand{\ul}[1]{\underline{#1}}
\newcommand{\C}{\mathbb{C}}
\newcommand{\spn}{\mathop{\mathrm{span}}}

%====header======
\newcommand{\solutions}[3]{
%\renewcommand{\thesolution}{{\large #2}.\arabic{problem}}
\vspace{-2ex}
\begin{center}
{\small  \courseNumber, \courseTitle
\hfill {\large \bf {Due: #1} }\\
\semester, University of Maryland \hfill
{\em Date: #3}}\\
\vspace{-1ex}
\hrulefill\\
\vspace{4ex}
{\Large #2}\\
\vspace{2ex}
\end{center}
\shortbar
\vspace{3ex}
}

\newcommand{\points}[1]{\textit{(#1 points)}}
\newcommand{\bonus}[1]{\textit{(Bonus: #1 points)}}

%    Q-circuit version 2
%    Copyright (C) 2004  Steve Flammia & Bryan Eastin
%    Last modified on: 9/16/2011
%
%    This program is free software; you can redistribute it and/or modify
%    it under the terms of the GNU General Public License as published by
%    the Free Software Foundation; either version 2 of the License, or
%    (at your option) any later version.
%
%    This program is distributed in the hope that it will be useful,
%    but WITHOUT ANY WARRANTY; without even the implied warranty of
%    MERCHANTABILITY or FITNESS FOR A PARTICULAR PURPOSE.  See the
%    GNU General Public License for more details.
%
%    You should have received a copy of the GNU General Public License
%    along with this program; if not, write to the Free Software
%    Foundation, Inc., 59 Temple Place, Suite 330, Boston, MA  02111-1307  USA

% Thanks to the Xy-pic guys, Kristoffer H Rose, Ross Moore, and Daniel Müllner,
% for their help in making Qcircuit work with Xy-pic version 3.8.  
% Thanks also to Dave Clader, Andrew Childs, Rafael Possignolo, Tyson Williams,
% Sergio Boixo, Cris Moore, Jonas Anderson, and Stephan Mertens for helping us test 
% and/or develop the new version.

\usepackage{xy}
\xyoption{matrix}
\xyoption{frame}
\xyoption{arrow}
\xyoption{arc}

\usepackage{ifpdf}
\ifpdf
\else
\PackageWarningNoLine{Qcircuit}{Qcircuit is loading in Postscript mode.  The Xy-pic options ps and dvips will be loaded.  If you wish to use other Postscript drivers for Xy-pic, you must modify the code in Qcircuit.tex}
%    The following options load the drivers most commonly required to
%    get proper Postscript output from Xy-pic.  Should these fail to work,
%    try replacing the following two lines with some of the other options
%    given in the Xy-pic reference manual.
\xyoption{ps}
\xyoption{dvips}
\fi

% The following resets Xy-pic matrix alignment to the pre-3.8 default, as
% required by Qcircuit.
\entrymodifiers={!C\entrybox}

\newcommand{\bra}[1]{{\left\langle{#1}\right\vert}}
\newcommand{\ket}[1]{{\left\vert{#1}\right\rangle}}
    % Defines Dirac notation. %7/5/07 added extra braces so that the commands will work in subscripts.
\newcommand{\qw}[1][-1]{\ar @{-} [0,#1]}
    % Defines a wire that connects horizontally.  By default it connects to the object on the left of the current object.
    % WARNING: Wire commands must appear after the gate in any given entry.
\newcommand{\qwx}[1][-1]{\ar @{-} [#1,0]}
    % Defines a wire that connects vertically.  By default it connects to the object above the current object.
    % WARNING: Wire commands must appear after the gate in any given entry.
\newcommand{\cw}[1][-1]{\ar @{=} [0,#1]}
    % Defines a classical wire that connects horizontally.  By default it connects to the object on the left of the current object.
    % WARNING: Wire commands must appear after the gate in any given entry.
\newcommand{\cwx}[1][-1]{\ar @{=} [#1,0]}
    % Defines a classical wire that connects vertically.  By default it connects to the object above the current object.
    % WARNING: Wire commands must appear after the gate in any given entry.
\newcommand{\gate}[1]{*+<.6em>{#1} \POS ="i","i"+UR;"i"+UL **\dir{-};"i"+DL **\dir{-};"i"+DR **\dir{-};"i"+UR **\dir{-},"i" \qw}
    % Boxes the argument, making a gate.
\newcommand{\meter}{*=<1.8em,1.4em>{\xy ="j","j"-<.778em,.322em>;{"j"+<.778em,-.322em> \ellipse ur,_{}},"j"-<0em,.4em>;p+<.5em,.9em> **\dir{-},"j"+<2.2em,2.2em>*{},"j"-<2.2em,2.2em>*{} \endxy} \POS ="i","i"+UR;"i"+UL **\dir{-};"i"+DL **\dir{-};"i"+DR **\dir{-};"i"+UR **\dir{-},"i" \qw}
    % Inserts a measurement meter.
    % In case you're wondering, the constants .778em and .322em specify
    % one quarter of a circle with radius 1.1em.
    % The points added at + and - <2.2em,2.2em> are there to strech the
    % canvas, ensuring that the size is unaffected by erratic spacing issues
    % with the arc.
\newcommand{\measure}[1]{*+[F-:<.9em>]{#1} \qw}
    % Inserts a measurement bubble with user defined text.
\newcommand{\measuretab}[1]{*{\xy*+<.6em>{#1}="e";"e"+UL;"e"+UR **\dir{-};"e"+DR **\dir{-};"e"+DL **\dir{-};"e"+LC-<.5em,0em> **\dir{-};"e"+UL **\dir{-} \endxy} \qw}
    % Inserts a measurement tab with user defined text.
\newcommand{\measureD}[1]{*{\xy*+=<0em,.1em>{#1}="e";"e"+UR+<0em,.25em>;"e"+UL+<-.5em,.25em> **\dir{-};"e"+DL+<-.5em,-.25em> **\dir{-};"e"+DR+<0em,-.25em> **\dir{-};{"e"+UR+<0em,.25em>\ellipse^{}};"e"+C:,+(0,1)*{} \endxy} \qw}
    % Inserts a D-shaped measurement gate with user defined text.
\newcommand{\multimeasure}[2]{*+<1em,.9em>{\hphantom{#2}} \qw \POS[0,0].[#1,0];p !C *{#2},p \drop\frm<.9em>{-}}
    % Draws a multiple qubit measurement bubble starting at the current position and spanning #1 additional gates below.
    % #2 gives the label for the gate.
    % You must use an argument of the same width as #2 in \ghost for the wires to connect properly on the lower lines.
\newcommand{\multimeasureD}[2]{*+<1em,.9em>{\hphantom{#2}} \POS [0,0]="i",[0,0].[#1,0]="e",!C *{#2},"e"+UR-<.8em,0em>;"e"+UL **\dir{-};"e"+DL **\dir{-};"e"+DR+<-.8em,0em> **\dir{-};{"e"+DR+<0em,.8em>\ellipse^{}};"e"+UR+<0em,-.8em> **\dir{-};{"e"+UR-<.8em,0em>\ellipse^{}},"i" \qw}
    % Draws a multiple qubit D-shaped measurement gate starting at the current position and spanning #1 additional gates below.
    % #2 gives the label for the gate.
    % You must use an argument of the same width as #2 in \ghost for the wires to connect properly on the lower lines.
\newcommand{\control}{*!<0em,.025em>-=-<.2em>{\bullet}}
    % Inserts an unconnected control.
\newcommand{\controlo}{*+<.01em>{\xy -<.095em>*\xycircle<.19em>{} \endxy}}
    % Inserts a unconnected control-on-0.
\newcommand{\ctrl}[1]{\control \qwx[#1] \qw}
    % Inserts a control and connects it to the object #1 wires below.
\newcommand{\ctrlo}[1]{\controlo \qwx[#1] \qw}
    % Inserts a control-on-0 and connects it to the object #1 wires below.
\newcommand{\targ}{*+<.02em,.02em>{\xy ="i","i"-<.39em,0em>;"i"+<.39em,0em> **\dir{-}, "i"-<0em,.39em>;"i"+<0em,.39em> **\dir{-},"i"*\xycircle<.4em>{} \endxy} \qw}
    % Inserts a CNOT target.
\newcommand{\qswap}{*=<0em>{\times} \qw}
    % Inserts half a swap gate.
    % Must be connected to the other swap with \qwx.
\newcommand{\multigate}[2]{*+<1em,.9em>{\hphantom{#2}} \POS [0,0]="i",[0,0].[#1,0]="e",!C *{#2},"e"+UR;"e"+UL **\dir{-};"e"+DL **\dir{-};"e"+DR **\dir{-};"e"+UR **\dir{-},"i" \qw}
    % Draws a multiple qubit gate starting at the current position and spanning #1 additional gates below.
    % #2 gives the label for the gate.
    % You must use an argument of the same width as #2 in \ghost for the wires to connect properly on the lower lines.
\newcommand{\ghost}[1]{*+<1em,.9em>{\hphantom{#1}} \qw}
    % Leaves space for \multigate on wires other than the one on which \multigate appears.  Without this command wires will cross your gate.
    % #1 should match the second argument in the corresponding \multigate.
\newcommand{\push}[1]{*{#1}}
    % Inserts #1, overriding the default that causes entries to have zero size.  This command takes the place of a gate.
    % Like a gate, it must precede any wire commands.
    % \push is useful for forcing columns apart.
    % NOTE: It might be useful to know that a gate is about 1.3 times the height of its contents.  I.e. \gate{M} is 1.3em tall.
    % WARNING: \push must appear before any wire commands and may not appear in an entry with a gate or label.
\newcommand{\gategroup}[6]{\POS"#1,#2"."#3,#2"."#1,#4"."#3,#4"!C*+<#5>\frm{#6}}
    % Constructs a box or bracket enclosing the square block spanning rows #1-#3 and columns=#2-#4.
    % The block is given a margin #5/2, so #5 should be a valid length.
    % #6 can take the following arguments -- or . or _\} or ^\} or \{ or \} or _) or ^) or ( or ) where the first two options yield dashed and
    % dotted boxes respectively, and the last eight options yield bottom, top, left, and right braces of the curly or normal variety.  See the Xy-pic reference manual for more options.
    % \gategroup can appear at the end of any gate entry, but it's good form to pick either the last entry or one of the corner gates.
    % BUG: \gategroup uses the four corner gates to determine the size of the bounding box.  Other gates may stick out of that box.  See \prop.

\newcommand{\rstick}[1]{*!L!<-.5em,0em>=<0em>{#1}}
    % Centers the left side of #1 in the cell.  Intended for lining up wire labels.  Note that non-gates have default size zero.
\newcommand{\lstick}[1]{*!R!<.5em,0em>=<0em>{#1}}
    % Centers the right side of #1 in the cell.  Intended for lining up wire labels.  Note that non-gates have default size zero.
\newcommand{\ustick}[1]{*!D!<0em,-.5em>=<0em>{#1}}
    % Centers the bottom of #1 in the cell.  Intended for lining up wire labels.  Note that non-gates have default size zero.
\newcommand{\dstick}[1]{*!U!<0em,.5em>=<0em>{#1}}
    % Centers the top of #1 in the cell.  Intended for lining up wire labels.  Note that non-gates have default size zero.
\newcommand{\Qcircuit}{\xymatrix @*=<0em>}
    % Defines \Qcircuit as an \xymatrix with entries of default size 0em.
\newcommand{\link}[2]{\ar @{-} [#1,#2]}
    % Draws a wire or connecting line to the element #1 rows down and #2 columns forward.
\newcommand{\pureghost}[1]{*+<1em,.9em>{\hphantom{#1}}}
    % Same as \ghost except it omits the wire leading to the left. 


\let\ket\undefined
\let\bra\undefined


\include{qmacros}
\usepackage{colonequals}
%\newcommand{\<}{\langle}
%\renewcommand{\>}{\rangle}

\begin{document}
%%%%%%%%%%%%%%%%%%%%%%%%%%%%%%%%%%%%%%%%%%%%%%%%%
\solutions{April 20th, 2023}{Assignment 4}{\today}
%%%%%%%%%%%%%%%%%%%%%%%%%%%%%%%%%%%%%%%%%%%%%%%%%
%
% Begin the solution for each problem by
% \begin{solution}{Problem Number} and ends it with \end{solution}
%
% the solution for Problem 1

Please submit it electronically to ELMS. This assignment is 6\% in your total points. For the simplicity
of the grading, the total points for the assignment are 60. Note that we will reward the use of Latex for
typesetting with bonus points (an extra 5\% of your points).

\begin{problem}{1}
\emph{The Fourier transform and translation invariance.}
The quantum Fourier transform on $n$ qubits is defined as the transformation
\[
  |x\> \mapsto \frac{1}{\sqrt{2^n}} \sum_{y=0}^{2^n-1} e^{2\pi i xy/2^n} |y\>
\]
where we identify $n$-bit strings and the integers they represent in binary.
More generally, for any nonnegative integer $N$, we can define the quantum Fourier transform modulo $N$ as
\[
  |x\> \stackrel{F_N}{\mapsto} \frac{1}{\sqrt{N}} \sum_{y=0}^{N-1} e^{2\pi i xy/N} |y\>
\]
where the state space is $\C^N$, with orthonormal basis $\{|0\>,|1\>,\ldots,|N-1\>\}$.
Let $P$ denote the unitary operation that adds $1$ modulo $N$: for any $x \in \{0,1,\ldots,N-1\}$, $P|x\>=|x+1 \bmod N\>$.
\begin{enumerate}
\item \points{2} Show that $F_N$ is a unitary transformation.
  \item \points{5} Show that the Fourier basis states are eigenvectors of $P$.  What are their eigenvalues?  (Equivalently, show that $F_N^{-1}PF_N$ is diagonal, and find its diagonal entries.)
  \item \points{3} Let $|\psi\>$ be a state of $n$ qubits.  Show that if $P|\psi\>$ is measured in the Fourier basis (or equivalently, if we apply the inverse Fourier transform and then measure in the computational basis), the probabilities of all measurement outcomes are the same as if the state had been $|\psi\>$.
\end{enumerate}
\end{problem}

\begin{problem}{2}
\emph{Factoring 21.}
\begin{enumerate}
  \item \points{3} Suppose that, when running Shor's algorithm to factor the number $21$, you choose the value $a=2$.
  What is the order $r$ of $a \bmod 21$?
  \item \points{3} Give an expression for the probabilities of the possible measurement outcomes when performing phase estimation with $n$ bits of precision in Shor's algorithm.
  \item \points{3} In the execution of Shor's algorithm considered in part (a), suppose you perform phase estimation with $n=7$ bits of precision.  Plot the probabilities of the possible measurement outcomes obtained by the algorithm.
You are encouraged to use software to produce your plot.
  \item \points{3} Compute $\gcd(21,a^{r/2}-1)$ and $\gcd(21,a^{r/2}+1)$.  How do they relate to the prime factors of $21$?
  \item \points{3} How would your above answers change if instead of taking $a=2$, you had taken $a=5$?
\end{enumerate}
\end{problem}

\begin{problem}{3}
\emph{Density matrices.}
Consider the ensemble in which the state $|0\>$ occurs with probability $3/5$ and the state $(|0\>+|1\>)/\sqrt2$ occurs with probability $2/5$.
\begin{enumerate}
  \item \points{2} What is the density matrix $\rho$ of this ensemble?
  \item \points{3} Write $\rho$ in the form $\frac{1}{2}(I + r_x X + r_y Y + r_z Z)$, and plot $\rho$ as a point in the Bloch sphere.
  \item \points{3} Suppose we measure the state in the computational basis.  What is the probability of getting the outcome $0$?  Compute this both by averaging over the ensemble of pure states and by computing $\tr(\rho|0\>\<0|)$, and show that the results are consistent.
  \item \points{3} How does the density matrix change if we apply the Hadamard gate?  Compute this both by applying the Hadamard gate to each pure state in the ensemble and finding the corresponding density matrix, and by computing $H \rho H^\dag$.
\end{enumerate}
\end{problem}

\begin{problem}{4}
\emph{Local operations and the partial trace.}
\begin{enumerate}
  \item \points{3} Let $|\psi\>=\frac{\sqrt 3}{2} |00\> + \frac{1}{2} |11\>$.  Let $\rho$ denote the density matrix of $|\psi\>$ and let $\rho'$ denote the density matrix of $(I \otimes H)|\psi\>$.  Compute $\rho$ and $\rho'$.
  \item \points{3} Compute $\tr_B(\rho)$ and $\tr_B(\rho')$, where $B$ refers to the second qubit.
  \item \points{4} Let $\rho$ be a density matrix for a quantum system with a bipartite state space $A \otimes B$.  Let $I$ denote the identity operation on system $A$, and let $U$ be a unitary operation on system $B$.  Prove that $\tr_{B}(\rho) = \tr_{B}((I \otimes U)\rho(I \otimes U^\dag))$.
  \item \points{3} Show that the converse of part (c) holds for pure states.  In other words, show that if $|\psi\>$ and $|\phi\>$ are bipartite pure states, and $\tr_{B}(|\psi\>\<\psi|) = \tr_{B}(|\phi\>\<\phi|)$, then there is a unitary operation $U$ acting on system $B$ such that $|\phi\> = (I \otimes U)|\psi\>$.
  \item \points{2} Does the converse of part (c) hold for general density matrices?  Prove or disprove it.
\end{enumerate}
\end{problem}

\begin{problem}{5} \emph{Product and entangled states.}  Determine which of the following states are entangled.  If the state is not entangled, show how to write it as a tensor product; if it is entangled, prove this.
\begin{enumerate}
  \item \points{3} $\frac{2}{3}|00\> + \frac{1}{3}|01\> - \frac{2}{3}|11\>$
  \item \points{3} $\frac{1}{2}(|00\>-i|01\>+i|10\>+|11\>)$
  \item \points{3} $\frac{1}{2}(|00\>-|01\>+|10\>+|11\>)$
\end{enumerate}
\end{problem}


\end{document}
