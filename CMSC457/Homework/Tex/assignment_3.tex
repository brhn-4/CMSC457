%You can leave alone everything before Line 84.
\documentclass{article}
\usepackage{url,amsfonts, amsmath, amssymb, amsthm}
\usepackage{pict2e}
\usepackage[percent]{overpic}
% Page layout
\setlength{\textheight}{8.75in}
\setlength{\columnsep}{2.0pc}
\setlength{\textwidth}{6.5in}
\setlength{\topmargin}{0in}
\setlength{\headheight}{0.0in}
\setlength{\headsep}{0.0in}
\setlength{\oddsidemargin}{0in}
\setlength{\evensidemargin}{0in}
\setlength{\parindent}{1pc}
\newcommand{\shortbar}{\begin{center}\rule{5ex}{0.1pt}\end{center}}
%\renewcommand{\baselinestretch}{1.1}
% Macros for course info
\newcommand{\courseNumber}{CMSC/PHYS 457}
\newcommand{\courseTitle}{Introduction to Quantum Computing}
\newcommand{\semester}{Spring 2023}
% Theorem-like structures are numbered within SECTION units
\theoremstyle{plain}
\newtheorem{theorem}{Theorem}[section]
\newtheorem{lemma}[theorem]{Lemma}
\newtheorem{corollary}[theorem]{Corollary}
\newtheorem{proposition}[theorem]{Proposition}
\newtheorem{statement}[theorem]{Statement}
\newtheorem{conjecture}[theorem]{Conjecture}
\newtheorem{fact}{Fact}
%definition style
\theoremstyle{definition}
\newtheorem{definition}[theorem]{Definition}
\newtheorem{example}{Example}
%\newtheorem{problem}{Problem}
\newtheorem{exercise}{Exercise}
\newtheorem{algorithm}{Algorithm}
%remark style
\theoremstyle{remark}
\newtheorem{remark}[theorem]{Remark}
\newtheorem{reduction}[theorem]{Reduction}
%\newtheorem{question}[theorem]{Question}
\newtheorem{question}{Question}
%\newtheorem{claim}[theorem]{Claim}
%
% Proof-making commands and environments
\newcommand{\beginproof}{\medskip\noindent{\bf Proof.~}}
\newcommand{\beginproofof}[1]{\medskip\noindent{\bf Proof of #1.~}}
\newcommand{\finishproof}{\hspace{0.2ex}\rule{1ex}{1ex}}
\newenvironment{problem}[1]{\medskip\noindent{\bf Problem #1.~}}{\shortbar}
\newenvironment{solution}[1]{\medskip\noindent{\bf Solution #1.~}}{\shortbar}
\newcommand{\ul}[1]{\underline{#1}}
\newcommand{\C}{\mathbb{C}}
\newcommand{\spn}{\mathop{\mathrm{span}}}

%====header======
\newcommand{\solutions}[3]{
%\renewcommand{\thesolution}{{\large #2}.\arabic{problem}}
\vspace{-2ex}
\begin{center}
{\small  \courseNumber, \courseTitle
\hfill {\large \bf {Due: #1} }\\
\semester, University of Maryland \hfill
{\em Date: #3}}\\
\vspace{-1ex}
\hrulefill\\
\vspace{4ex}
{\Large #2}\\
\vspace{2ex}
\end{center}
\shortbar
\vspace{3ex}
}

\newcommand{\points}[1]{\textit{(#1 points)}}
\newcommand{\bonus}[1]{\textit{(Bonus: #1 points)}}

%    Q-circuit version 2
%    Copyright (C) 2004  Steve Flammia & Bryan Eastin
%    Last modified on: 9/16/2011
%
%    This program is free software; you can redistribute it and/or modify
%    it under the terms of the GNU General Public License as published by
%    the Free Software Foundation; either version 2 of the License, or
%    (at your option) any later version.
%
%    This program is distributed in the hope that it will be useful,
%    but WITHOUT ANY WARRANTY; without even the implied warranty of
%    MERCHANTABILITY or FITNESS FOR A PARTICULAR PURPOSE.  See the
%    GNU General Public License for more details.
%
%    You should have received a copy of the GNU General Public License
%    along with this program; if not, write to the Free Software
%    Foundation, Inc., 59 Temple Place, Suite 330, Boston, MA  02111-1307  USA

% Thanks to the Xy-pic guys, Kristoffer H Rose, Ross Moore, and Daniel Müllner,
% for their help in making Qcircuit work with Xy-pic version 3.8.  
% Thanks also to Dave Clader, Andrew Childs, Rafael Possignolo, Tyson Williams,
% Sergio Boixo, Cris Moore, Jonas Anderson, and Stephan Mertens for helping us test 
% and/or develop the new version.

\usepackage{xy}
\xyoption{matrix}
\xyoption{frame}
\xyoption{arrow}
\xyoption{arc}

\usepackage{ifpdf}
\ifpdf
\else
\PackageWarningNoLine{Qcircuit}{Qcircuit is loading in Postscript mode.  The Xy-pic options ps and dvips will be loaded.  If you wish to use other Postscript drivers for Xy-pic, you must modify the code in Qcircuit.tex}
%    The following options load the drivers most commonly required to
%    get proper Postscript output from Xy-pic.  Should these fail to work,
%    try replacing the following two lines with some of the other options
%    given in the Xy-pic reference manual.
\xyoption{ps}
\xyoption{dvips}
\fi

% The following resets Xy-pic matrix alignment to the pre-3.8 default, as
% required by Qcircuit.
\entrymodifiers={!C\entrybox}

\newcommand{\bra}[1]{{\left\langle{#1}\right\vert}}
\newcommand{\ket}[1]{{\left\vert{#1}\right\rangle}}
    % Defines Dirac notation. %7/5/07 added extra braces so that the commands will work in subscripts.
\newcommand{\qw}[1][-1]{\ar @{-} [0,#1]}
    % Defines a wire that connects horizontally.  By default it connects to the object on the left of the current object.
    % WARNING: Wire commands must appear after the gate in any given entry.
\newcommand{\qwx}[1][-1]{\ar @{-} [#1,0]}
    % Defines a wire that connects vertically.  By default it connects to the object above the current object.
    % WARNING: Wire commands must appear after the gate in any given entry.
\newcommand{\cw}[1][-1]{\ar @{=} [0,#1]}
    % Defines a classical wire that connects horizontally.  By default it connects to the object on the left of the current object.
    % WARNING: Wire commands must appear after the gate in any given entry.
\newcommand{\cwx}[1][-1]{\ar @{=} [#1,0]}
    % Defines a classical wire that connects vertically.  By default it connects to the object above the current object.
    % WARNING: Wire commands must appear after the gate in any given entry.
\newcommand{\gate}[1]{*+<.6em>{#1} \POS ="i","i"+UR;"i"+UL **\dir{-};"i"+DL **\dir{-};"i"+DR **\dir{-};"i"+UR **\dir{-},"i" \qw}
    % Boxes the argument, making a gate.
\newcommand{\meter}{*=<1.8em,1.4em>{\xy ="j","j"-<.778em,.322em>;{"j"+<.778em,-.322em> \ellipse ur,_{}},"j"-<0em,.4em>;p+<.5em,.9em> **\dir{-},"j"+<2.2em,2.2em>*{},"j"-<2.2em,2.2em>*{} \endxy} \POS ="i","i"+UR;"i"+UL **\dir{-};"i"+DL **\dir{-};"i"+DR **\dir{-};"i"+UR **\dir{-},"i" \qw}
    % Inserts a measurement meter.
    % In case you're wondering, the constants .778em and .322em specify
    % one quarter of a circle with radius 1.1em.
    % The points added at + and - <2.2em,2.2em> are there to strech the
    % canvas, ensuring that the size is unaffected by erratic spacing issues
    % with the arc.
\newcommand{\measure}[1]{*+[F-:<.9em>]{#1} \qw}
    % Inserts a measurement bubble with user defined text.
\newcommand{\measuretab}[1]{*{\xy*+<.6em>{#1}="e";"e"+UL;"e"+UR **\dir{-};"e"+DR **\dir{-};"e"+DL **\dir{-};"e"+LC-<.5em,0em> **\dir{-};"e"+UL **\dir{-} \endxy} \qw}
    % Inserts a measurement tab with user defined text.
\newcommand{\measureD}[1]{*{\xy*+=<0em,.1em>{#1}="e";"e"+UR+<0em,.25em>;"e"+UL+<-.5em,.25em> **\dir{-};"e"+DL+<-.5em,-.25em> **\dir{-};"e"+DR+<0em,-.25em> **\dir{-};{"e"+UR+<0em,.25em>\ellipse^{}};"e"+C:,+(0,1)*{} \endxy} \qw}
    % Inserts a D-shaped measurement gate with user defined text.
\newcommand{\multimeasure}[2]{*+<1em,.9em>{\hphantom{#2}} \qw \POS[0,0].[#1,0];p !C *{#2},p \drop\frm<.9em>{-}}
    % Draws a multiple qubit measurement bubble starting at the current position and spanning #1 additional gates below.
    % #2 gives the label for the gate.
    % You must use an argument of the same width as #2 in \ghost for the wires to connect properly on the lower lines.
\newcommand{\multimeasureD}[2]{*+<1em,.9em>{\hphantom{#2}} \POS [0,0]="i",[0,0].[#1,0]="e",!C *{#2},"e"+UR-<.8em,0em>;"e"+UL **\dir{-};"e"+DL **\dir{-};"e"+DR+<-.8em,0em> **\dir{-};{"e"+DR+<0em,.8em>\ellipse^{}};"e"+UR+<0em,-.8em> **\dir{-};{"e"+UR-<.8em,0em>\ellipse^{}},"i" \qw}
    % Draws a multiple qubit D-shaped measurement gate starting at the current position and spanning #1 additional gates below.
    % #2 gives the label for the gate.
    % You must use an argument of the same width as #2 in \ghost for the wires to connect properly on the lower lines.
\newcommand{\control}{*!<0em,.025em>-=-<.2em>{\bullet}}
    % Inserts an unconnected control.
\newcommand{\controlo}{*+<.01em>{\xy -<.095em>*\xycircle<.19em>{} \endxy}}
    % Inserts a unconnected control-on-0.
\newcommand{\ctrl}[1]{\control \qwx[#1] \qw}
    % Inserts a control and connects it to the object #1 wires below.
\newcommand{\ctrlo}[1]{\controlo \qwx[#1] \qw}
    % Inserts a control-on-0 and connects it to the object #1 wires below.
\newcommand{\targ}{*+<.02em,.02em>{\xy ="i","i"-<.39em,0em>;"i"+<.39em,0em> **\dir{-}, "i"-<0em,.39em>;"i"+<0em,.39em> **\dir{-},"i"*\xycircle<.4em>{} \endxy} \qw}
    % Inserts a CNOT target.
\newcommand{\qswap}{*=<0em>{\times} \qw}
    % Inserts half a swap gate.
    % Must be connected to the other swap with \qwx.
\newcommand{\multigate}[2]{*+<1em,.9em>{\hphantom{#2}} \POS [0,0]="i",[0,0].[#1,0]="e",!C *{#2},"e"+UR;"e"+UL **\dir{-};"e"+DL **\dir{-};"e"+DR **\dir{-};"e"+UR **\dir{-},"i" \qw}
    % Draws a multiple qubit gate starting at the current position and spanning #1 additional gates below.
    % #2 gives the label for the gate.
    % You must use an argument of the same width as #2 in \ghost for the wires to connect properly on the lower lines.
\newcommand{\ghost}[1]{*+<1em,.9em>{\hphantom{#1}} \qw}
    % Leaves space for \multigate on wires other than the one on which \multigate appears.  Without this command wires will cross your gate.
    % #1 should match the second argument in the corresponding \multigate.
\newcommand{\push}[1]{*{#1}}
    % Inserts #1, overriding the default that causes entries to have zero size.  This command takes the place of a gate.
    % Like a gate, it must precede any wire commands.
    % \push is useful for forcing columns apart.
    % NOTE: It might be useful to know that a gate is about 1.3 times the height of its contents.  I.e. \gate{M} is 1.3em tall.
    % WARNING: \push must appear before any wire commands and may not appear in an entry with a gate or label.
\newcommand{\gategroup}[6]{\POS"#1,#2"."#3,#2"."#1,#4"."#3,#4"!C*+<#5>\frm{#6}}
    % Constructs a box or bracket enclosing the square block spanning rows #1-#3 and columns=#2-#4.
    % The block is given a margin #5/2, so #5 should be a valid length.
    % #6 can take the following arguments -- or . or _\} or ^\} or \{ or \} or _) or ^) or ( or ) where the first two options yield dashed and
    % dotted boxes respectively, and the last eight options yield bottom, top, left, and right braces of the curly or normal variety.  See the Xy-pic reference manual for more options.
    % \gategroup can appear at the end of any gate entry, but it's good form to pick either the last entry or one of the corner gates.
    % BUG: \gategroup uses the four corner gates to determine the size of the bounding box.  Other gates may stick out of that box.  See \prop.

\newcommand{\rstick}[1]{*!L!<-.5em,0em>=<0em>{#1}}
    % Centers the left side of #1 in the cell.  Intended for lining up wire labels.  Note that non-gates have default size zero.
\newcommand{\lstick}[1]{*!R!<.5em,0em>=<0em>{#1}}
    % Centers the right side of #1 in the cell.  Intended for lining up wire labels.  Note that non-gates have default size zero.
\newcommand{\ustick}[1]{*!D!<0em,-.5em>=<0em>{#1}}
    % Centers the bottom of #1 in the cell.  Intended for lining up wire labels.  Note that non-gates have default size zero.
\newcommand{\dstick}[1]{*!U!<0em,.5em>=<0em>{#1}}
    % Centers the top of #1 in the cell.  Intended for lining up wire labels.  Note that non-gates have default size zero.
\newcommand{\Qcircuit}{\xymatrix @*=<0em>}
    % Defines \Qcircuit as an \xymatrix with entries of default size 0em.
\newcommand{\link}[2]{\ar @{-} [#1,#2]}
    % Draws a wire or connecting line to the element #1 rows down and #2 columns forward.
\newcommand{\pureghost}[1]{*+<1em,.9em>{\hphantom{#1}}}
    % Same as \ghost except it omits the wire leading to the left. 


\let\ket\undefined
\let\bra\undefined

\include{qmacros}
\usepackage{colonequals}
%\newcommand{\<}{\langle}
%\renewcommand{\>}{\rangle}

\begin{document}
%%%%%%%%%%%%%%%%%%%%%%%%%%%%%%%%%%%%%%%%%%%%%%%%%
\solutions{April 6th, 2023}{Assignment 3}{\today}
%%%%%%%%%%%%%%%%%%%%%%%%%%%%%%%%%%%%%%%%%%%%%%%%%
%
% Begin the solution for each problem by
% \begin{solution}{Problem Number} and ends it with \end{solution}
%
% the solution for Problem 1

Please submit it electronically to ELMS. This assignment is 6\% in your total points. For the simplicity
of the grading, the total points for the assignment are 60. Note that we will reward the use of Latex for
typesetting with bonus points (an extra 5\% of your points).

\begin{problem}{1}
\emph{The Bernstein-Vazirani problem.}

\begin{enumerate}
  \item \points{3} Suppose $f:\{0,1\}^n \to \{0,1\}$ is a function of the form
  \[ f(\ul{x})=x_1s_1 + x_2s_2 + \cdots + x_n s_n \bmod 2 \] for some unknown $\ul{s} \in \{0,1\}^n$.  Given a black box for $f$, how many classical queries are required to learn $s$ with certainty?
  \item \points{4} Prove that for any $n$-bit string $\ul{u} \in \{0,1\}^n$,
  \[ \sum_{\ul{v} \in \{0,1\}^n} (-1)^{\ul{u}\cdot\ul{v}} = \begin{cases}2^n & \text{if $\ul{u}=\ul{0}$} \\ 0 & \text{otherwise} \end{cases} \]
  where $\ul{0}$ denotes the $n$-bit string $00\ldots0$.
  \item \points{4} Let $U_f$ denote a quantum black box for $f$, acting as
  $U_f|\ul{x}\>|y\> = |\ul{x}\>|y \oplus f(\ul{x})\>$
  for any $\ul{x} \in \{0,1\}^n$ and $y \in \{0,1\}$.
  Show that the output of the following circuit is the state $|\ul{s}\> (|0\>-|1\>)/\sqrt{2}$.
  \[
  \Qcircuit @C=1em @R=.7em @!R {
  \lstick{|0\>} & \gate{H} & \multigate{4}{U_f} & \gate{H} & \qw \\
  \lstick{|0\>} & \gate{H} & \ghost{U_f} & \gate{H} & \qw \\
  \lstick{\vdots\,\,\,} & \vdots &   & \vdots \\
  \lstick{|0\>} & \gate{H} & \ghost{U_f} & \gate{H} & \qw \\
  \lstick{\frac{|0\>-|1\>}{\sqrt2}} & \qw & \ghost{U_f} & \qw & \qw
  }
  \]
   \item \points{1} What can you conclude about the quantum query complexity of learning $s$?
\end{enumerate}
\end{problem}

\begin{problem}{2}
 \emph{Determining the "slope" of a linear function over $\mathbb{Z}_4$}. 
Let $\mathbb{Z}_4 = \{0, 1, 2, 3\}$, with arithmetic operations of addition and multiplication defined with respect to modulo 4 arithmetic on this set. Suppose that we are given a black-box computing a linear function $f: \mathbb{Z}_4 \rightarrow \mathbb{Z}_4$, which of the form $f(x) = ax+b$, with unknown coefficients $a, b \in \mathbb{Z}_4$ (throughout this question, multiplication and addition mean these operations in modulo 4 arithmetic). Let our goal be to determine the coefficient $a$ (the "slope" of the function). We will consider the number of quantum and classical queries needed to solve this problem.

Assume that what we are given is a black box for the function $f$ that is in reversible form in the following sense. For each $x,y \in \mathbb{Z}_4$, the black box maps $(x,y)$ to $(x,y + f(x))$ in the classical case; and $\ket{x}\ket{y}$ to $\ket{x} \ket{y + f (x)}$ in the quantum case (which is unitary).

Also, note that we can encode the elements of $\mathbb{Z}_4$ into 2-bit strings, using the usual representation of integers as a binary strings (00 = 0, 01 = 1, 10 = 2, 11 = 3). With this encoding, we can view $f$ as a function on 2-bit strings $f : \{0,1\}^2 \rightarrow \{0, 1\}^2$. When refering to the elements of $\mathbb{Z}_4$, we use the notation $\{0, 1, 2, 3\}$ and $\{00, 01, 10, 11\}$ interchangeably.

\begin{itemize}
  \item[(1)] \points{5} Prove that every classical algorithm for solving this problem must make two queries.
  \item[(2)] \points{5} Consider the 2-qubit unitary operation $A$ corresponding to "add 1", such that $A\ket{x} =
\ket{x+1}$ for all $x \in \mathbb{Z}_4$. It is easy to check that 
   \[ A= \left( \begin{array}{cccc}
0 & 0 & 0 & 1\\
1 & 0 & 0 & 0 \\
0 & 1 & 0 & 0\\
0 & 0 & 1 & 0 \end{array} \right).\] 
Let $\ket{\psi} =\frac{1}{2} (\ket{00} + i \ket{01}+ i^2 \ket{10} +i^3 \ket{11})$, where $i=\sqrt{-1}$. Prove that $A\ket{\psi}=-i\ket{\psi}$.   
  \item[(3)] \points{5} Show how to create the state $\frac{1}{2}((-i)^{f(00)}\ket{00}+(-i)^{f(01)}\ket{01}+(-i)^{f(10)}\ket{10}+(-i)^{f(11)}\ket{11})$ with a single query to $U_f$. (Hint: you may use the result in part (2) for this.)
  \item[(4)] \points{5} Show how to solve the problem (i.e., determine the coefficient $a \in \mathbb{Z}_4$) with a single
quantum query to $f$. (Hint: you may use the result in part (3) for this.)
\end{itemize}
\end{problem}

\begin{problem}{3} \emph{Simon's algorithm and its extension.}
In Simon's problem, recall that we're given oracle access to a
function $f : \{0, 1\}^n \rightarrow \{0, 1\}^n$ with the promise that there exists a secret string $s \neq 0^n$
such that $f(x) = f(y)$ if and only if $y = x \oplus s$ for all different $x, y\in \{0, 1\}^n$. 

\begin{enumerate}
   \item \points{5} Recall the algorithm described during the lecture. Rigorously prove that $O(n)$ repetitions of Simon's
algorithm are enough if we want to succeed with $1-e^{-n}$ probability.
   \item \points{10} Suppose instead that there are two nonzero secret strings, $s\neq t$, such that $f(x) =
f(x \oplus s) = f(x \oplus  t) = f(x \oplus s \oplus t)$ for all x. Describe a variation of Simon's algorithm
that finds the entire set $s, t, s\oplus t$ in time polynomial in $n$. When you measure a state in
your algorithm, what are the possible results of the measurement? How do you use those
measurement results to reconstruct the set $s, t, s\oplus t$?
\end{enumerate}
\end{problem}


\begin{problem}{4}
\emph{Searching for a quantum state}.

Suppose you are given a black box $U_\phi$ that identifies an unknown quantum state $|\phi\>$ (which may not be a computational basis state).  Specifically, $U_\phi |\phi\> = -|\phi\>$, and $U_\phi|\xi\>=|\xi\>$ for any state $|\xi\>$ satisfying $\<\phi|\xi\>=0$.

Consider an algorithm for preparing $|\phi\>$ that starts from some fixed state $|\psi\>$ and repeatedly applies the unitary transformation $VU_\phi$, where $V=2|\psi\>\<\psi|-I$ is a reflection about $|\psi\>$.

Let $|\phi^\perp\> = \frac{e^{-i\lambda}|\psi\> - \sin(\theta) |\phi\>}{\cos(\theta)}$ denote a state orthogonal to $|\phi\>$ in $\spn\{|\phi\>,|\psi\>\}$, where $\<\phi|\psi\> = e^{i\lambda} \sin(\theta)$ for some $\lambda,\theta \in \R$.

\begin{enumerate}
  \item \points{2} Write the initial state $|\psi\>$ in the basis $\{|\phi\>,|\phi^\perp\>\}$.
  \item \points{3} Write $U_\phi$ and $V$ as matrices in the basis $\{|\phi\>,|\phi^\perp\>\}$.
  \item \points{3} Let $k$ be a positive integer.  Compute $(VU_\phi)^k$. 
  \item \points{3} Compute $\<\phi|(VU_\phi)^k|\psi\>$.
  \item \points{2} Suppose that $|\<\phi|\psi\>|$ is small.  Approximately what value of $k$ should you choose in order for the algorithm to prepare a state close to $|\phi\>$, up to a global phase?  Express your answer in terms of $|\<\phi|\psi\>|$.
\end{enumerate}
\end{problem}

\end{document}
